\documentclass{article}
\usepackage{array}

\begin{document}

\begin{table}[ht]
\centering
\begin{tabular}{|>{\raggedright\arraybackslash}p{1.5cm}|>{\raggedright\arraybackslash}p{3.5cm}|>{\raggedright\arraybackslash}p{7cm}|}
\hline
\textbf{Value} & \textbf{Description} & \textbf{Motivation from Simulation Perspective} \\
\hline
'U' & Uninitialized & Indicates that the signal has not been initialized, helping detect uninitialized signals during simulation. \\
\hline
'X' & Forcing Unknown & Represents an unknown state due to signal conflicts or undefined values, useful for detecting driver conflicts. \\
\hline
'0' & Logic 0 & Represents logical low (false), typically used for normal binary operation. \\
\hline
'1' & Logic 1 & Represents logical high (true), typically used for normal binary operation. \\
\hline
'Z' & High Impedance & Represents a tri-state condition, useful for modeling buses and tri-state buffers where signals are not driven. \\
\hline
'W' & Weak Unknown & Indicates a weakly driven unknown state, useful for modeling situations where the drive strength is insufficient. \\
\hline
'L' & Weak 0 & Represents a weakly driven logic 0, commonly used for weak pull-down configurations. \\
\hline
'H' & Weak 1 & Represents a weakly driven logic 1, commonly used for weak pull-up configurations. \\
\hline
'-' & Don't Care & Represents a "don't care" condition, useful for synthesis optimization where the exact value is irrelevant. \\
\hline
\end{tabular}
\caption{Possible Values of the \texttt{std\_logic} Type and Their Simulation Motivations}
\end{table}

\end{document}